\chapter{Introduction}

    The Norwegian \gls{imr} is owned by the the Ministry of Fisheries and Coastal Affairs and its role is to perform research and act as a advisory service in questions surrounding marine ecosystems and aquaculture\cite{IMR}. Fishing quotas is one such advisory task and includes the gathering of large amounts of acoustic data. This data is collected by research vessels\cite{IMR-vessels} equipped with an array of up to five echosounders (18kHz, 38kHz, 70kHz, 120kHz and 200kHz), research equipment, trawl equipment and facilities for the crew. This quickly becomes an expensive endeavour and now the \gls{imr} is looking into alternative solutions to acquire the same data. 
    
    Lightweight autonomous drones\cite{johnsen2020measuring} has been tested in Årdalsfjorden and involved a kayak with a electric motor and only one 200kHz echosounder installed. The drone were enabled for autonomous behavior as well as direct remote control. Results showed that the drone were able to measure fish that usually live closer to the surface in the acoustic 8 meter blind zone of the larger vessels as the echosounders are installed on the bottom part of the research ship. In addition the drone produces significantly less noise in the water and are less prone to scare away the fish. The lightweight platform enabled the kayak-drone to collect data from shallower water earlier inaccessible to the larger vessels. The end result was a success for the kayak-drone, but the the manned vessels are still needed for the biological samples in addition to the acoustic data. 
    
    
    

\section{Research questions}
    \begin{itemize}
        \item \textbf{As the lightweight vessels do not have the structural capacity to carry all the five echosounders the \gls{imr} usually deploy, which ones should be prioritized?}
        
        My work will use data focused around sand eel, hence the results for other species will most likely differ.

\end{itemize}


\subsection{Listings}
You can do listings, like in Listing~\ref{ListingReference}
\begin{lstlisting}[caption={[Short caption]Look at this cool listing. Find the rest in Appendix~\ref{Listing}},label=ListingReference]
$ java -jar myAwesomeCode.jar
\end{lstlisting}

You can also do language highlighting for instance with Golang:
And in line~\ref{LineThatDoesSomething} of Listing~\ref{ListingGolang} you can see that we can ref to lines in listings.

\begin{lstlisting}[caption={Hello world in Golang},label=ListingGolang,escapechar=|]
package main

import "fmt"

func main() {
    fmt.Println("hello world") |\label{LineThatDoesSomething}|
}

\end{lstlisting}

\subsection{Figures}

Example of a centred figure
\begin{figure}[H]
    \centering
    \includegraphics[scale=0.5]{figures/Flowchart}
    \caption{Caption for flowchart}
  	\medskip 
	\hspace*{15pt}\hbox{\scriptsize Credit: Acme company makes everything \url{https://acme.com/}}
    \label{FlowchartFigure}
\end{figure}

\subsection{Tables}

We can also do tables. Protip: use \url{https://www.tablesgenerator.com/} for generating tables.
\begin{table}[H]
\centering
\caption{Caption of table}
\label{TableLabel}
\begin{tabular}{|l|l|l|}
\hline
Title1 & Title2 & Title3 \\ \hline
data1  & data2  & data3  \\ \hline
\end{tabular}
\end{table}

\subsection{}