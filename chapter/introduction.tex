\chapter{Introduction}


    \section{Problem}
    
    
    
    The Norwegian \gls{imr} is owned by the Ministry of Fisheries and Coastal Affairs, and its role is to perform research and act as an advisory service in questions surrounding marine ecosystems and aquaculture\cite{IMR}. Fishing quotas is one such advisory task and includes the gathering of large amounts of acoustic data. This data is collected by research vessels\cite{IMR-vessels} equipped with an array of up to six echosounders, research equipment, trawl equipment and facilities for the crew. This quickly becomes an expensive endeavour and now the \gls{imr} is looking into alternative solutions to acquire the same data. 
    
    Lightweight autonomous drones\cite{johnsen2020measuring} has been tested in Årdalsfjorden and involved a kayak with an electric motor and only one 200kHz echosounder installed. The drone were enabled for autonomous behavior as well as direct remote control. Results showed that the drone were able to measure fish that usually live closer to the surface in the acoustic 8-meter blind zone of the larger vessels, as the echosounders are installed on the bottom part of the research ship. In addition, the drone produces significantly less noise in the water and are less prone to scare away the fish. The lightweight platform enabled the kayak-drone to collect data from shallower water, earlier inaccessible to the larger vessels. The end result was a success for the kayak-drone, but the manned vessels are still needed for the biological samples in addition to the acoustic data. 
    
    %\section{Difference picture and acoustic data}
%\todo{DO THIS}
    

    %\textbf{\gls{cv}} is a very popular field of research within deep learning\cite{voulodimos2018deep_computer_vision}. This is because it has mainly focused on tasks we humans do seemingly without training, while computers struggle with it. Examples are facial recognition and object detection in video and image data. There are also works that have gone beyond human capability, like \citeauthor{davis2014visual_deep_video_audio}s\cite{davis2014visual_deep_video_audio} work, where they recovered sounds from the vibrations they induced in objects captured on video. 
    
    
    
    
    \section{Relevance}
        things

\section{Research Questions}
    \begin{itemize}
        \item \textbf{As a lightweight vessel do not have the capacity to carry all the six echo sounders the \gls{imr} usually deploy, which ones should be prioritized when classifying sandeel in multi-frequency acoustic data?}
        
        My work will use data focused around sand eel, hence the results for other species will most likely differ.
        \item Mulige:
        \item Kan man trene en model med god ytelse ved å bruke pseudo labels fra en annen model?
        \item 
    \end{itemize}



%You can do listings, like in Listing~\ref{ListingReference}
%\begin{lstlisting}[caption={[Short caption]Look at this cool listing. Find the rest in %Appendix~\ref{Listing}},label=ListingReference]
%$ java -jar myAwesomeCode.jar
%\end{lstlisting}

You can also do language highlighting for instance with Golang:
And in line~\ref{LineThatDoesSomething} of Listing~\ref{ListingGolang} you can see that we can ref to lines in listings.

\begin{lstlisting}[caption={Hello world in Golang},label=ListingGolang,escapechar=|]
package main

import "fmt"

func main() {
    fmt.Println("hello world") |\label{LineThatDoesSomething}|
}

\end{lstlisting}



Example of a centred figure
\begin{figure}[H]
    \centering
    \includegraphics[scale=0.5]{figures/Flowchart}
    \caption{Caption for flowchart}
  	\medskip 
	\hspace*{15pt}\hbox{\scriptsize Credit: Acme company makes everything \url{https://acme.com/}}
    \label{FlowchartFigure}
\end{figure}



