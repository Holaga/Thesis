\chapter{Conclusion and Future Work}
    In this thesis, acoustic classifiers were successfully trained on pseudo labels with varying subsets of frequencies. The performance of each subset was measured to advice on how to equip lightweight, unnamed vessels with echo sounders to monitor sandeel. Based on our results, one can conclude that the information gained when selecting the maximally informative frequencies subsets increase from a size of one up to two, and drastically to three. However, from a size of three and afterward the subsets depend on the frequencies \textit{18kHz}, \textit{38kHz}, and \textit{200kHz} to maximize information. Considering the size of transducers operating at these frequencies, the unmanned kayak introduced in section \ref{Unmanned Vehicles in Marine Science} is likely too small to carry this subset and larger unnmanned vessels should be considered when classifying sandeel.
    
    %\todo{Based on these simulations we conclude that the relative frequency response has a
%greater impact on the model’s predictions than the shape of the fish schools.}
    
    %This was done by training a series of models with the same architecture on labels produced by an already existing acoustic sandeel classifier, where each model had different subsets of the frequencies available during training.
    
    %We sought the most informative subset of frequencies to be utilized when classifying sandeel in acoustic data. This was done to help identify an optimal choice of frequencies, if the choice of transducers were restricted by for example price or carrying capacity in autonomous vessels. To measure the information lost, we started by producing pseudo labels with an existing automatic acoustic classifier trained to identify sandeel. Then we trained new classifiers based on the same architecture on the same data, but varying subsets of frequencies were used. We could then measure how well these new models could reconstruct the pseudo labels. To find and rank the best performing subset of frequencies, we conducted a greedy search on each subset size used during training. We found that the F1-score increased from a single frequency to two, and then drastically to three, after which only marginal improvements were seen. In particular, the subset containing \textit{18kHz}, \textit{38kHz}, and \textit{200kHz} achieved close to the same performance as using the complete set of six frequencies. Furthermore, the three frequencies mentioned exhibit unique performance compared to the other subsets of equal size. 

    
     In future research, we propose that the experiment created in this thesis should be tested and trained on additional data and with new features. Only the data from 2018 was used as test data during this work, and our results should be verified across years. Especially those years used in the basis work of \citeauthor{brautaset2020acoustic}, focusing on verifying the performance of subset: \textit{18kHz}, \textit{38kHz}, and \textit{200kHz}. The method implemented to create the pseudo labels can be further optimized and new avenues within \gls{kd} should be tested. Additionally, we propose the implementation of a "depth in water" feature in our model to provide the \gls{cnn} with additional context for each crop.
    
    %As the \textit{background} include unknown backscatter, further work should identify if more labels could be drawn from this class and be available to the model. This could give a more exact performance in regard to sandeel, but depends on which labels are produced when the operators generate the data.
    
    %- more data
    %- actual labels, alter the training scheme of original model
    %- continue training of single frequency classifiers
    %- bias to planktop and other unknown effects, more classes and have fewer uknown things in the ignore / background class.
    
 