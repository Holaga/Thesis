\chapter{The data and tools}
    The data was provided by the \gls{imr} and I had access to a broad selection of yearly trawl cruises spanning from 2011 to 2020. In this chapter, I will delve into the data itself to give you a clearer picture of how I looks, how I acquired it and also how I preprocessed it to be readable by Python.
    
    \section{Windows Azure}
        Windows Azure\cite{azure} is a platform owned by Microsoft that provides cloud solutions for several services. My use was to access a remote storage provided by the \gls{imr} and mount this to my local computer. Thus enabling me to download the data for this thesis. 
    
    \section{The data}
        The files I downloaded came to me as ".RAW" \cite{raw} and ".WORK" files. ".RAW" is the uncompressed raw output from the echo sounder. The same format is used by cameras before they are converted to, for example, ".JPEG". Because they are uncompressed, they do not lose any data. Unfortunately, this also makes them large. The ".WORK" files are the annotations of the ".RAW" files done by operators using a system called the \Gls{lsss}\cite{lsss}. 
        
        The data from 2020 which spanned three months took 240 GB of storage space. 
        
        SJEKK DATAEKSEMPEL!

    \section{Docker}
        Docker\cite{docker} is an open source platform that provides what they call containerization and is owned by the company under the same name, Docker, Inc. If you are familiar with virtual machines, then containerization will be very familiar to you. Docker is based on the Linux kernel, and enables you to create a container, which is an independent process that uses resources from the main instance. For each container, you can manage its own dependencies like programming languages and libraries. These containers can then be shared with others as images files, and as they can be run without the receiver having to manage the aforementioned dependencies as this is built into the image. Thus, you can make an application or code very easily accessible for other people, as long as they have installed Docker.
        
    \section{Zarr}
        By using the ".ZARR"\cite{zarr} format, you gain access to store chunked compressed  multidimensional arrays. There are several highlights from this library, but I used primarily the ability to access the arrays on disk. This means I did not need to load the entire array into memory and could work with the array and access parts of it without hardware limitations.
        
        
    \section{CRIMAC-pipeline}
        Together, the \gls{nr} and the \gls{imr} developed a pipeline\cite{crimac_pipeline} to classify the acoustic backscatter in echo sounder data. This was part of the \gls{crimac} project and is accessed by using docker. They also provide the image for a container to access the data through Azure.
        
        The pipeline is run using one docker container, which in turn downloads and runs four others:
        
            \begin{description}
              \item[$\bullet$ Preprocessor] Preprocesses the ".RAW" and ".WORK" to respectively ".ZARR" and ".PARQUET" files. Since frequencies can have different resolutions, all are re-gridded to the 38kHz frequency.
              \item[$\bullet$ Unet] Using a pretrained deep learning model called Unet it produces pixel based annotations
              \item[$\bullet$ Bottom detection] Identifies the bottom and generates a pixel based map stored as ".ZARR".
              \item[$\bullet$ Report generation] Takes in the output from the bottom detection, Unet and the preprocessed ".RAW" file and generates a report for the \gls{ices}.
    
            \end{description}
        
        
        
        
        
        The Unet, bottom detection and preprocessing is the same as described in section \ref{unet_paper_acoustic}.
        

        
        
    \section{Xarray}
        Xarray\cite{xarray} is a Python package that is made for working with multidimensional arrays. It is based on NumPy and adds labels in the form of attributes and coordinates on top of the NumPy-arrays. This was the library I used for accessing and working more efficiently with the ".ZARR" arrays, as I was already very familiar with NumPy.
        
        
    \section{The final data}
        I followed the above sections almost sequentially for acquiring my data as Python compatible. By using docker, I downloaded the raw data using Azure and ran individual parts of the CRIMAC-pipeline as these could be run independently. This meant I did not use the report generator. Then I accessed the files using Xarray and started doing my own preprocessing of the data.
        
        HERE COME MORE OF THE CHECKS FOR BOTTOM, SANDEEL and stuff.
        

        

        