\chapter{Related work}



\section{Paper 1: Acoustic classification in multifrequency echosounder data using deep convolutional neural networks} \label{unet_paper_acoustic}
    
    
    OBS! Må ha forklart U-net, accoustic klassifikasjon, F1 OBS, feature construction, statistical and machine learning methods!
    
    To help assess the size of legal fishing quotas, acoustic trawls surveys are initiated to gather data. Before the data is of use, an operator manually needs to interpret the data in a time-consuming process to assign the acoustic back scattering to the correct category, but in turn often introduces bias. Several methods have been implemented to reduce this bias and with a goal to automate the process, but they all have a common weakness where they need a predefined feature space.  In 2020 a group of researchers implemented a U-net deep learning model to the problem and as this is an \gls{cnn} it does not need to have pre-designed features but learn them from the data. Sandeel  ~\cite{brautaset2020acoustic}. 
    
    
    
    \subsection{Result}
        things
    \subsection{Problems}