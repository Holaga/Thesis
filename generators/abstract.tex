\pagenumbering{roman}

\begin{abstract} 

\noindent     In this thesis we sought the most informative subset of frequencies to be utilized when classifying sandeel in acoustic data. This was to help identify an optimal choice of frequencies, if the choice of transducers were restricted by for example price or carrying capacity in autonomous vessels. To measure the information lost, we started by producing pseudo labels with an existing automatic acoustic classifier trained to identify sandeel. Then we trained new classifiers based on the same architecture on the same data, but varying subsets of frequencies were used. We could then measure how well these new models could reconstruct the pseudo labels.  The  F1-score of the highest performing subset, per subset size, increased from a size of one frequency in use (0.34) to two (0.46), and then drastically to three (0.65), after which only marginal improvements were seen. In particular, the subset containing \textit{18kHz}, \textit{38kHz}, and \textit{200kHz} achieved close to the same performance as using the complete set of six frequencies (0.67). Furthermore, the three frequencies mentioned exhibit unique performance compared to the other subsets of equal size. 

\end{abstract}

\renewcommand{\abstractname}{Acknowledgements}
\begin{abstract}
	
	
	\vspace{1cm}
	\hspace*{\fill}\texttt{Knut Thormod Aarnes Holager}\\ 
	\hspace*{\fill} 01 June, 2022
\end{abstract}
\newpage