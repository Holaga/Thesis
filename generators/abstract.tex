\pagenumbering{roman}

\begin{abstract} 

\noindent     In this thesis, we sought the most informative subset of frequencies to be utilized when classifying sandeel in acoustic data. The intention was to help identify an optimal choice of frequencies if the choice of transducers were restricted by, for example, price or carrying capacity in autonomous vessels. To measure the information lost, we started by producing pseudo labels with an existing automatic acoustic classifier trained to identify sandeel. Then we trained new classifiers based on the same architecture on the same data, but varying subsets of frequencies were used. We could then measure how well these new models could reconstruct the pseudo labels.  The  F1-score of the highest performing subset, per subset size, increased from a size of one frequency in use (0.34) to two (0.46), and then drastically to three (0.65), after which only marginal improvements were seen. In particular, the subset containing \textit{18kHz}, \textit{38kHz}, and \textit{200kHz} achieved close to the same performance as using the complete set of six frequencies (0.67). Furthermore, the three frequencies mentioned exhibit unique performance compared to the other subsets of equal size. 

\end{abstract}

\renewcommand{\abstractname}{Acknowledgements}
\begin{abstract}
	I would foremost like to express my gratitude towards my friends and fellow master's students. Hans Martin Johansen, Mathias Madslien, Halvor Barndon, Emir Zamwa, Johanna Jøsang, John Isak Villanger, and Gunnar. Thanks to you all for the comradeship, both in the reading room and outside the university. For making each day of this degree enjoyable and supporting me throughout these two years. Particularly to my fellow sandwich makers, whose ignorance of conventional mayonnaise limits, motivated me to continually hit the gym. \newline

     Thanks to my supervisor Ketil Malde and co-supervisor Nils Olav Handegard for helping me formulate my thesis and providing invaluable feedback. Gratitude to all the people who provided support from the University of Bergen and IMR, particularly Ibrahim Umar and Tomasz Furmanek, for without your expert knowledge and willingness to help, this thesis would not have been possible.  Finally, a big thanks to my family and girlfriend for your unfaltering support. 

	
	\vspace{1cm}
	\hspace*{\fill}\texttt{Knut Thormod Aarnes Holager}\\ 
	\hspace*{\fill} 01 June, 2022
\end{abstract}
\newpage