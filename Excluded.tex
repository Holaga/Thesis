    %Through this chapter, I will introduce the setting of my work and my data, which in turn leads to the data preparation, and how my experiments were to be conducted. 
    
    %\section{The setting} \label{setting}
        %The work was conducted over a two-year period from August 2020 to June 2022 at the University of Bergen. My task was provided by the \gls{imr} and so I was invited to regular meetings with their machine learning group as well as the \gls{crimac} group. The initial task developed from an interest in the individual frequency performance of the frequencies involved in the work described in section \ref{unet_paper_acoustic} and I were to base the work I did on their work. Initially, access was granted by the University, to a local server I will call \textit{janus} but later to a new more powerful server called \textit{birget}. These servers were intended to be the locations to run the heavy machine learning algorithms, as these needs to run for long spans of time and had the storage capacity for the data. 
        
        

%\subsection{Restrictions}
            %Here I will explain three major restrictions which had a heavy impact on the task and is essential to understand my approach to this task.
            %\subsubsection{The .zarr format issue}
               % It was decided that my task was to use the \textit{.zarr} format, which was the new standard format that the \gls{imr} were to use to handle acoustic \textit{.raw} data. This proved later to be a problem, as the pipeline \gls{crimac} had developed were yet to be compatible with .zarr formatted input data for training new models. The training of new models, as described later in the experiments section, was crucial for my work. The \gls{imr} could not provide a time for when this feature would be developed, and the problem was discovered close to the end of the third semester out of four total semesters. With the remaining time starting to become an issue, it was decided to exclude the training part of the \gls{crimac} pipeline, and solve the task with a different approach and is described in section \ref{Pseudo label}. Furthermore, the complete pipeline could now not be utilized.
            
             %\subsubsection{The .work bitmap issue}\label{bitmap_error}
                %Earlier, the \textit{.work} file contain the annotations from the operators and would translate to a bitmap for the labels per class. However, these files were found to be full of errors, different formats for storing the labels, or in some cases missing, and caused the \gls{crimac} preprocessor module to crash or give erroneous outputs in the \textit{.parquet} file. This issue had no certain date to be repaired, and appeared at the same time as the \textit{.zarr} training issue was evaluated, and so was deemed not working when planning my further work. This is also why they are not visualized is the illustrations.

            %\subsubsection{Servers at the University of Bergen}
                %We began to use the server \textit{janus} already during the first semester of the masters program. Over time, we discovered that this server did not have the capacity to facilitate more than some 3-4 students at once and had regular severe issues. The server saw use of professors from the institute and more than 10 students that I personally knew of. This caused heavy restrictions in the processing power afforded to each user and set unfortunate boundaries on our work. 
                
                %In late January of the last semester of the masters program, we gained access to a powerful new server named \textit{birget} and this server could meet the needs of my work. However, this was far too late for major parts of my work, as many already had been implemented with the boundaries set by janus. This heavily affected choices for my method, and in particular the amount of data I could use, as the process of acquiring and processing this data was computationally expensive and had to be conducted on janus as birget lacked the software needed.
                
                
%        This would mean that the labels would consist of predictions produced from the four frequencies used in article \ref{unet_paper_acoustic}, namely; 18kHz, 38kHz, 120kHz and 200kHz. These frequencies were also present in my data. These output of the from the \gls{crimac} U-Net were a segmentation map of probabilities in the range [0,1], and so I applied a threshold of 0.8 to convert the values below to 0 and above to 1. Hence, producing a mask for each class \textit{sandeel}, \textit{other} and \textit{background}, and could be used as labels.